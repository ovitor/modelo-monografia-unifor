\chapter{Introdução}\label{cap:introducao}

De todo o volume d'àgua do planeta terra, 97,5\% são salgadas e 2,493\% de água
doce, mas que estão concentrados em geleiras ou regiões subterrâneas de difícil
acesso; portanto, apenas 0,007\% é doce e está acessível para o uso humano,
disponível em rios, lagos e na atmosfera (SHIKLOMANOV, 1998). Ainda que o Brasil
seja privilegiado por possuir 12\% da água doce do planeta, sua distribuição não
é favorável para os seres vivos todas regiões do país, já que a maioria dessa
quantidade está na Amazônia.

\begin{citacao} 
Esta omissão deve ser entendida no contexto mundial da época,
muito diferente do atual; o colonialismo seguia sendo uma força dominante e
muitos dos países cujas populações sofriam por falta de acesso à água e ao
saneamento não estavam presentes na mesa de negociação. Naquele momento, a
sociedade civil não desempenhava um papel tão notório como na atualidade,
chamando nossa atenção e a dos nossos governos sobre o sofrimento das pessoas no
mundo. Os países apresentavam um menor grau de urbanização, com um reduzido
número de assentamentos informais densamente povoados, o que significava que o
problema da falta de água e saneamento das zonas urbanas não era tão extremo
como é na atualidade (ALBUQUERQUE; ROAF, 2012, p. 29-30, traduziu-se).
\end{citacao}

\section{Motivação para a Dissertação}\label{sec:motivacao}
\lipsum[5-6]
\lipsum[1-2]
\lipsum[3-4]

\section{Descrição do problema}\label{sec:descricao-problema}
\lipsum[3-4]
\lipsum[1-2]
\lipsum[3-4]
\lipsum[1-2]

\section{Objetivos Geral e Específicos}\label{sec:objetivos}
\lipsum[1]

Em seu sentido mais particular, os seguintes objetivos específicos são:

\begin{itemize}
	\item Excepteur sint occaecat cupidatat non proident, sunt in culpa qui officia deserunt mollit anim id est laborum;
	\item Lorem ipsum dolor sit amet, consectetur adipisicing elit, sed do eiusmod;
	\item Tempor incididunt ut labore et dolore magna aliqua. Ut enim ad minim veniam;
	\item Quis nostrud exercitation ullamco laboris nisi ut aliquip ex ea commodo consequat. Duis aute irure dolor in reprehenderit in voluptate velit esse;
	\item Cillum dolore eu fugiat nulla pariatur. Excepteur sint occaecat cupidatat non.
\end{itemize}

\section{Produção científica}\label{sec:producao}
Durante este projeto de mestrado, os seguintes trabalhos científicos foram aceitos e publicados, a saber:

\begin{itemize}
	\item \textbf{Einstein, A.}, 1905. \textbf{The photoelectric effect}. Ann. Phys, 17(132), p.4;
	\item \textbf{Einstein, A.}, 1904. \textbf{Zur allgemeinen molekularen Theorie der Wärme}. Annalen der Physik, 319(7), pp.354-362.
\end{itemize}

\section{Estrutura da Dissertação}\label{sec:estrutura}
\lipsum[4-5]
\lipsum[1-2]
\lipsum[1-4]
\lipsum[3-2]