%% abtex2-modelo-trabalho-academico.tex, v<VERSION> laurocesar
%% Copyright 2012-<COPYRIGHT_YEAR> by abnTeX2 group at http://www.abntex.net.br/
%%
%% This work may be distributed and/or modified under the
%% conditions of the LaTeX Project Public License, either version 1.3
%% of this license or (at your option) any later version.
%% The latest version of this license is in
%%   http://www.latex-project.org/lppl.txt
%% and version 1.3 or later is part of all distributions of LaTeX
%% version 2005/12/01 or later.
%%
%% This work has the LPPL maintenance status `maintained'.
%%
%% The Current Maintainer of this work is the abnTeX2 team, led
%% by Lauro César Araujo. Further information are available on
%% http://www.abntex.net.br/
%%
%% This work consists of the files abntex2-modelo-trabalho-academico.tex,
%% abntex2-modelo-include-comandos and abntex2-modelo-references.bib
%%

% ------------------------------------------------------------------------
% ------------------------------------------------------------------------
% abnTeX2: Modelo de Trabalho Academico (tese de doutorado, dissertacao de
% mestrado e trabalhos monograficos em geral) em conformidade com
% ABNT NBR 14724:2011: Informacao e documentacao - Trabalhos academicos -
% Apresentacao
% ------------------------------------------------------------------------
% ------------------------------------------------------------------------

\documentclass[
	12pt,				% tamanho da fonte
	openright,			% capítulos começam em pág ímpar (insere página vazia caso preciso)
	twoside,			% para impressão em recto e verso. Oposto a oneside
	a4paper,			% tamanho do papel.
	% -- opções da classe abntex2 --
	chapter=TITLE,		% títulos de capítulos convertidos em letras maiúsculas
	%section=TITLE,		% títulos de seções convertidos em letras maiúsculas
	%subsection=TITLE,	% títulos de subseções convertidos em letras maiúsculas
	%subsubsection=TITLE,% títulos de subsubseções convertidos em letras maiúsculas
	% -- opções do pacote babel --
	english,			% idioma adicional para hifenização
	french,				% idioma adicional para hifenização
	spanish,			% idioma adicional para hifenização
	brazil				% o último idioma é o principal do documento
	]{abntex2}

% ---
% Pacotes básicos
% ---
\usepackage{lmodern}			% Usa a fonte Latin Modern
\usepackage[utf8]{inputenc}		% Codificacao do documento (conversão automática dos acentos)
\usepackage[T1]{fontenc}		% Selecao de codigos de fonte.
\usepackage{lastpage}			% Usado pela Ficha catalográfica
\usepackage{indentfirst}		% Indenta o primeiro parágrafo de cada seção.
\usepackage{color}				% Controle das cores
\usepackage{graphicx}			% Inclusão de gráficos
\usepackage{float}				% Habilitar opção H nos elementos
\usepackage{microtype} 			% para melhorias de justificação
\usepackage{mathptmx}
\usepackage{utils/modelo} 					% Pacote com informações da customização
\usepackage{utils/facilitadores} 					% Pacote com facilitadore 
\usepackage{xstring} 									% Criar comandos com IF
\usepackage{caption}
\usepackage{subcaption}
\usepackage{lipsum}				% para geração de dummy text

% ---
% Pacotes de citações
% ---
\usepackage[brazilian,hyperpageref]{backref}	 % Paginas com as citações na bibl
\usepackage[alf]{abntex2cite}	% Citações padrão ABNT

% ---
% CONFIGURAÇÕES DE PACOTES
% ---
\setlist[itemize]{noitemsep, topsep=0pt, leftmargin=1.75cm}
\graphicspath{{figuras/imagens/}}  % Inclusão dos paths para imagens
\OnehalfSpace

% adicionando introducao ao sumario
%\newcommand*{\SkipTocEntry}{\addtocontents{toc}{\gobblefive}}
%\addtocontents{toc}{\protect\setcounter{tocdepth}{0}}

% ---
% Configurações do pacote backref
% Usado sem a opção hyperpageref de backref
\renewcommand{\backrefpagesname}{Citado na(s) página(s):~}
% Texto padrão antes do número das páginas
\renewcommand{\backref}{}
% Define os textos da citação
\renewcommand*{\backrefalt}[4]{
	\ifcase #1 %
		Nenhuma citação no texto.%
	\or
		Citado na página #2.%
	\else
		Citado #1 vezes nas páginas #2.%
	\fi}%
% ---

% ---
% Informações de dados para CAPA e FOLHA DE ROSTO
% ---
% Informações do Autor e do trabalho
% ------------------------------------
\autor{<Nome do Autor>}
\titulo{<Título da Monografia>}
\linha{<Linha de Pesquisa>} % Inteligência Artificial, Computação aplicada ou Engenharia de Software
\matricula{<Matrícula do Aluno>}
\orientador{<Nome do Orientador>}
\coorientador{<Nome do Coorientador>} % Se você tem um coorientador, descomente esta linha

% Professores convidados para a banca
% ------------------------------------
% 
%  - Caso tenha um terceiro professor convidado, remova os comentários

\nomeprofessorA{<Nome do Professor A>}
\instituicaoprofessorA{<Instituição do Professor A> (<Sigla A>)}

\nomeprofessorB{<Nome do Professor B>}
\instituicaoprofessorB{<Instituição do Professor B> (<Sigla B>)}

%\nomeprofessorC{<Nome do Professor C>}
%\instituicaoprofessorC{<Instituição do Professor C> (<Sigla C>)}

% ---
% Configurações de aparência do PDF final
% ---

% alterando o aspecto da cor azul
\definecolor{blue}{RGB}{0,0,0}

% informações do PDF
\makeatletter
\hypersetup{
     	%pagebackref=true,
		pdftitle={\@title},
		pdfauthor={\@author},
    	pdfsubject={\imprimirpreambulo},
	    pdfcreator={LaTeX with abnTeX2},
		pdfkeywords={abnt}{latex}{abntex}{abntex2}{trabalho acadêmico},
		colorlinks=true,       		% false: boxed links; true: colored links
    	linkcolor=blue,          	% color of internal links
    	citecolor=blue,        		% color of links to bibliography
    	filecolor=magenta,      		% color of file links
		urlcolor=blue,
		bookmarksdepth=4
}
\makeatother
% ---

% ---
% Espaçamentos entre linhas e parágrafos
% ---

% O tamanho do parágrafo é dado por:
\setlength{\parindent}{1.3cm}

% Controle do espaçamento entre um parágrafo e outro:
\setlength{\parskip}{0.4cm}  % tente também \onelineskip

% ---
% compila o indice
% ---
\makeindex
% ---

% ----
% Início do documento
% ----
\begin{document}

% Seleciona o idioma do documento (conforme pacotes do babel)
%\selectlanguage{english}
\selectlanguage{brazil}

% Retira espaço extra obsoleto entre as frases.
\frenchspacing

% ----------------------------------------------------------
% ELEMENTOS PRÉ-TEXTUAIS
% ----------------------------------------------------------
% \pretextual
\imprimircapa
\imprimirfolhaderosto 

\begin{folhadeaprovacao}

    \begin{center}
      \MakeUppercase{\large\imprimirautor}

      \vspace{1.5cm}
      \textbf{\MakeUppercase{\Large\imprimirtitulo}}
   \end{center}

   \vfill

   \hspace{.5\textwidth}
   \begin{minipage}{.4\textwidth}
    Monografia apresentada à banca examinadora e à Coordenação do Curso de
    Direito do Centro de Ciências Jurídicas da Universidade de Fortaleza, 
    adequada e aprovada para suprir exigência parcial inerente à obtenção
    do grau de bacharel em Direito, em conformidade com os normativos do MEC,
    regulamentada pela Res. nº R028/99 da Universidade de Fortaleza.
    \end{minipage}

    \vspace{1.5cm}

    {\imprimirlocal}, {\imprimirdata}.

    \vspace{1cm}

    \begin{minipage}{10cm}
     \imprimirorientador \\
     Prof. orientador da Universidade de Fortaleza

     \vspace{.5cm}

     \nomeprofessorA \\
     Prof. examinador da Universidade de Fortaleza

     \vspace{.5cm}

     \nomeprofessorB \\
     Prof. examinador da Universidade de Fortaleza

     \vspace{.5cm}

     \imprimircoorientador \\
     Prof. orientador de Metodologia

     \vspace{.5cm}

     Coordenação do Curso de Direito

    \end{minipage}

  \vfill
  
\end{folhadeaprovacao}


%\includepdf{pretextual/folhadeaprovacao.pdf} 	% Folha de aprovação (Depois da apresentação)
\begin{dedicatoria}
   \vspace*{\fill}
   \noindent
   \begin{flushright}
   \textit{ Este trabalho é dedicado às crianças adultas que,\\
   quando pequenas, sonharam em se tornar cientistas.} 
   \end{flushright}
\end{dedicatoria} 				% Dedicatória
\begin{agradecimentos}

Agradeço àquele que fez esse modelo.

\end{agradecimentos} 			% Agradecimento
% \begin{epigrafe}
    \vspace*{\fill}
	\begin{flushright}
		\textit{%
		``<Citação Célebre>''\\
		(<Autor da citação>)}
	\end{flushright}
\end{epigrafe} 				% Epígrafe
\setlength{\absparsep}{18pt} % ajusta o espaçamento dos parágrafos do resumo
\begin{resumo}
 Segundo a \citeonline[3.1-3.2]{NBR6028:2003}, o resumo deve ressaltar o
 objetivo, o método, os resultados e as conclusões do documento. A ordem e a extensão
 destes itens dependem do tipo de resumo (informativo ou indicativo) e do
 tratamento que cada item recebe no documento original. O resumo deve ser
 precedido da referência do documento, com exceção do resumo inserido no
 próprio documento. (\ldots) As palavras-chave devem figurar logo abaixo do
 resumo, antecedidas da expressão Palavras-chave:, separadas entre si por
 ponto e finalizadas também por ponto.

 \textbf{Palavras-chave}: latex. abntex. editoração de texto.
\end{resumo} 					% Resumo em português
% \input{pretextual/abstract.tex} 				% Resumo em inglês

% inserir lista de ilustrações
\pdfbookmark[0]{\listfigurename}{lof}
\listoffigures*
\cleardoublepage
% inserir lista de tabelas
\pdfbookmark[0]{\listtablename}{lot}
\listoftables*
\cleardoublepage
% ---
\begin{siglas}
  \item[ABNT] Associação Brasileira de Normas Técnicas
  \item[abnTeX] ABsurdas Normas para TeX
\end{siglas}  			% Lista de abreviaturas e siglas
% \begin{simbolos}
\item[$ \Gamma $] Letra grega Gama
\item[$ \Lambda $] Lambda
\item[$ \zeta $] Letra grega minúscula zeta
\item[$ \in $] Pertence
\end{simbolos} 				% Lista de símbolos
% inserir o sumario
\pdfbookmark[0]{\contentsname}{toc}
\tableofcontents*
\cleardoublepage
% ---

% ----------------------------------------------------------
% ELEMENTOS TEXTUAIS
% ----------------------------------------------------------
\textual

%\addcontentsline{toc}{chapter}{INTRODUÇÃO}

\chapter{Introdução}
\label{cap:introducao}

\lipsum[1-2]
\lipsum[5-6]
\lipsum[1-2]

 				% Capítulo de introdução
\chapter{Lorem ipsum dolor sit amet}\label{cap:exampleChapter}
\thispagestyle{empty}

\lipsum[1]
% ---
\section{Aliquam vestibulum fringilla lorem}
% ---

\lipsum[1]

\figurasimples[talbot2012]{logo-unifor}{Exemplo de figura com fonte.}{4cm}

% ------
\lipsum[1]

\subsection{Aliquam vestibulum fringilla lorem}

\figurasimples{logo-unifor}{Exemplo de figura sem fonte.}{4cm}

\lipsum[1]

\tabela{tabela-exemplo}{Lista de bases de dados usados nesse trabalho}{1} % Exemplo de inclusão de tabela

\lipsum[1]

\lipsum[1]
% Exemplo de inclusão de gráficos

\figuradupla{pid}{Legenda da imagem 1}{rip}{Legenda da imagem 2}
% ------
\lipsum[2]
 			% Capitulo de exemplo

% ----------------------------------------------------------
% Finaliza a parte no bookmark do PDF
% para que se inicie o bookmark na raiz
% e adiciona espaço de parte no Sumário
% ----------------------------------------------------------
\phantompart

% ---
% Conclusão
% ---
\chapter{Conclusão}
% ---

\lipsum[31-33] 				% Capítulo de introdução

% ----------------------------------------------------------
% ELEMENTOS PÓS-TEXTUAIS
% ----------------------------------------------------------
\postextual
% ----------------------------------------------------------

% ----------------------------------------------------------
% Referências bibliográficas
% ----------------------------------------------------------
\bibliography{referencias}

% ----------------------------------------------------------
% Glossário
% ----------------------------------------------------------
%
% Consulte o manual da classe abntex2 para orientações sobre o glossário.
%
%\glossary

% ----------------------------------------------------------
% Apêndices
% ----------------------------------------------------------

% ---
% Inicia os apêndices
% ---
\begin{apendicesenv}

% Imprime uma página indicando o início dos apêndices
\partapendices
% Insere os apêndices
% ----------------------------------------------------------
\chapter{Nullam elementum urna vel imperdiet sodales elit ipsum pharetra ligula
ac pretium ante justo a nulla curabitur tristique arcu eu metus}
% ----------------------------------------------------------
\lipsum[55-57]
%\chapter{Quisque libero justo}
% ----------------------------------------------------------

\lipsum[50]

\end{apendicesenv}
% ---

% ----------------------------------------------------------
% Anexos
% ----------------------------------------------------------

% ---
% Inicia os anexos
% ---
\begin{anexosenv}

% Imprime uma página indicando o início dos anexos
\partanexos
% Insere os anexos
\chapter{Morbi ultrices rutrum lorem.}
% ---
\lipsum[30]

%\chapter{Cras non urna sed feugiat cum sociis natoque penatibus et magnis dis
parturient montes nascetur ridiculus mus}
% ---

\lipsum[31]

\end{anexosenv}

%---------------------------------------------------------------------
% INDICE REMISSIVO
%---------------------------------------------------------------------
\phantompart
\printindex
%---------------------------------------------------------------------

\end{document}
